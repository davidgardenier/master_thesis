%*******************************************************
% Abstract
%*******************************************************
%\renewcommand{\abstractname}{Abstract}
\pdfbookmark[1]{Abstract}{Abstract}
\begingroup
\let\clearpage\relax
\let\cleardoublepage\relax
\let\cleardoublepage\relax

\chapter*{Abstract}
Using Rossi X-ray Timing Explorer observations, a population study is conducted on the timing properties of both neutron star and black hole X-ray binaries. The ratios of integrated power in four equally spaced Fourier bands allows power spectral shapes to be parametrised with two 'power colour' values. These allow the evolution in timing properties across observations to be determined. Neutron star X-ray binaries are shown to follow a similar power spectral evolution to black hole X-ray binaries, confirming timing properties to have a common origin in the accretion disk. Power colour tracks show the type of compact object can easily be distinguished for X-ray binaries observations in the hard state. Evidence is found for a link between power colours and canonical neutron star spectral states, allowing a relationship to be established between black hole and neutron star states. \\

The influence of various systems parameters on the variability of neutron star observations have also been studied, such as the relation between the binary orbit inclination of neutron star X-ray binaries and their timing properties. While tentative signs of an inclination dependent track emerges upon comparing the hardness of an energy spectrum with power colours, no conclusive results can be drawn. The distribution of power colours for a number of subclassifications such as atoll and Z sources, as well as Sco- and Cyg-like sources were also studied. Investigating whether the type of compact object had any influence on the timing properties showed tentative evidence for changes in power colours due to strong magnetic fields. Power colours could provide a strong analysis method by which observations of both neutron star and black hole X-ray binaries can be classified according to type, state and potentially also inclination, allowing systems with similar geometries to be studied.

\newpage
\let\clearpage\relax
\let\cleardoublepage\relax
\let\cleardoublepage\relax
\pdfbookmark[1]{Popular Summary}{Popular Summary}
\chapter*{Popular Summary}
It is generally agreed that the universe is a mind-bogglingly vast place. As such, bungee-jumping into a volcano which is erupting could be seen as a fairly humdrum activity on scale of activities in extreme environments in the universe. If standing in the microwave meal section in the local supermarket is on one end of the scale, the other end would be finding yourself close to a compact object. Unsurprisingly perhaps, compact objects are both a) very compact, and b) an object. Much like Neapolitan ice cream, compact objects come in three flavours - white dwarfs, neutron stars and black holes. However much unlike Neapolitan ice cream, these objects float around in space, and in somewhat of a reversal of roles tend to eat, or accrete, from their surroundings. The Universe as a whole can be thankful they aren't fussy eaters: as of yet, no alphabet pasta or even Star Wars themed chocolate flavoured cereal shapes have been found in space. Instead, compact objects feed on stars - the nuclear fireballs providing warmth to planets and jobs for astronomers. As amusing as it would be if compact objects ate stars like Pacman eats his pills, reality tells a different story. These objects spiral around each other, the compact object savouring the star by slowly feeding off its outer layers. This food/matter falls into a spiral around the compact object before being devoured. In the process, it creates an accretion disk surrounding the compact object, which could perhaps be best described as having the form of a well-squished doughnut. In this work, I study compare such systems with each other, hoping to learn more about accretion onto compact objects. \\

So how can you study these systems? There are two common ways in which research is conducted in astronomy: you can take a single photo of a region of the sky and describe which objects you see, or you can take multiple photos over time and explain the evolution of these objects. As the type of systems I study are quite small in astronomical terms, we have to use timing information to probe the geometries of these systems. Imagine that suddenly a rather large piece of matter fell into the outer edges of the accretion disc; then it would take time before that increase in matter had spiralled its way into the inner regions. Assume now that the amount of mass roughly couples to the luminosity, or intensity of that region. As the outer parts of the accretion disk emit at lower energies than the inner parts, you expect to see a rise in intensity at lower energies before observing a rise at higher energies. Using such logic, you can deduce many properties of these systems from timing information.\\

In this study, I used a technique to reduce all the timing information gathered over the course of a single observation into just two parameters. Conducting an analysis of around 15000 observations allowed me to study how these parameters evolved for different systems. This indirectly tells us how the geometry and other intrinsic properties change over time. I found that systems with a neutron star evolved in a similar fashion to black hole systems, which confirms that the variability we see in the emission of both of these systems is primarily due changes in the accretion flow, rather than changes due having different compact objects.  I also found that a slightly change in the two timing parameters allows you to rapidly classify which type of compact object a system has. I compared the evolution in timing properties to other information such as the inclination of systems, and various subclassification schemes. While some signs were found pointing towards tentative relationships, further research would need to be done to conclusively prove or disprove these correlations. The results of my study are encouraging, as they show that the use of these timing parameters gives greater insight into the evolution and properties of these systems.


\endgroup