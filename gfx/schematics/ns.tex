\begin{landscape}
\begin{figure}[H]
\vspace*{-5cm}
\hspace*{0.5cm}
\captionsetup[subfloat]{justification=centering}
\subfloat{%
  \begin{minipage}{0.48\textwidth}
	\begin{tikzpicture}[xscale=2.5,yscale=1.25]

		\coordinate (zero) at (-2,-2);
		\coordinate[right=2 of zero] (a1);
		\draw[very thick,->] (zero) ++ (-0.15,0) -- (2,-2) node[below left] {log E};
		\draw[very thick,->] (zero) ++ (0,-0.15) -- (-2,2) node[above left,rotate=90] {log E$\cdot$F};
		\node[draw,thick,circle,fill=white] at (-2,-2) {A};
		\clip (-2,-2) rectangle (2,2);
		\begin{scope}[shift={(0,0.3)},yscale=1.2]
		%\draw[step=1cm,gray,very thin] (-2,-2) grid (2,2);
		\draw[scale=0.5,domain=-2:3,smooth,variable=\x,thick,dashed] plot ({\x*0.8-3.5},{-\x*\x+0.5});
		\node[] at (-1.55,-0.2) {\footnotesize BB};
		\draw[scale=0.5,domain=-2.5:2.5,smooth,variable=\x,thick,dashdotted] plot ({\x*0.05-1},{-\x*\x-2});
		\node[] at (-0.35,-1.7) {\footnotesize FL};
		\draw[scale=0.5,domain=-2.5:3,smooth,variable=\x,thick,dotted] plot ({\x*1.5-1.75},{-\x*\x-1});
		\node[] at (-0.5,-0.4) {\footnotesize BLBB};
		%\draw[scale=0.5,domain=-10:10,smooth,variable=\x,thick,solid] plot ({\x},{-0.5*\x-2.5});
		\draw  plot[thick,smooth, tension=.7] coordinates {(-1.8,-2.5) (-1.4,-0.9) (0.6,-1.9) (1,-2.5)};
		\node[] at (-1.6,-1.6) {\footnotesize PL};
		\end{scope}
	\draw[thick]  plot[smooth, tension=.7] coordinates {(-2,1.4) (-1.7,1.7) (-1.2,0.9) (-0.6,0.6) (-0.5,0.8) (-0.4,0.4)  (0.3,-0.5) (0.8,-2)};

		\node[draw,thick,circle,fill=white] at (-2,-2) {A};
\end{tikzpicture}
\label{fig:ns_sp}
  \end{minipage}}
\hspace{6cm} % induce horizontal separation
\subfloat{%
  \begin{minipage}[c]{0.48\textwidth}
	\begin{tikzpicture}[xscale=2.5,yscale=1.25]
		\coordinate (zero) at (-2,-2);
		\coordinate[right=2 of zero] (a1);
		\begin{scope}[shift={(0,-0.25)}]
		\draw[thick] (0,1.75) ellipse (1.5 and 0.45); %EIS
		\node[] at (0,1.75) {\footnotesize EIS};
		\draw[thick,xshift=-1.5cm,yshift=-1cm] plot [smooth cycle, tension=0.5] coordinates {(0,0.5) (0,2) (1.5,2)}; %IS
		\node[] at (-1,0.5) {\footnotesize IS};
		\draw[thick,rotate=-30,xshift=0.5cm,yshift=-.4cm] (-1,-1) ellipse (0.5 and 0.28);
		\node[] at (-1.125,-0.95) {\footnotesize LLB};
		\draw[thick] (0,-1.25) ellipse (0.6 and 0.25);
		\node[] at (0,-1.25) {\footnotesize LB};
		\draw[thick,rotate around={30:(1.2,-1)}] (1.2,-1) ellipse (0.5 and 0.25);
		\node[] at (1.2,-1) {\footnotesize UB};
		\end{scope}
		\draw[very thick,->] (zero) ++ (-0.15,0) -- (2,-2) node[below left] {Softness};
		\draw[very thick,->] (zero) ++ (0,-0.15) -- (-2,2) node[above left,rotate=90] {Hardness};
		\node[draw,thick,circle,fill=white] at (-2,-2) {B};
	\end{tikzpicture}
\label{fig:ns_atoll}
  \end{minipage}}

\hspace*{0.5cm}
\subfloat{%
  \begin{minipage}{0.48\textwidth}  
	\begin{tikzpicture}[xscale=2.5,yscale=1.25]
		\coordinate (zero) at (-2,-2);
		\coordinate[right=2 of zero] (a1);
		\begin{scope}[shift={(-0.5,-0.25)},yscale=1.8]
		\draw[thick,rotate around={-50:(-0.7,0.7)}] (-0.7,0.7) ellipse (0.45 and 0.15);
		\node[] at (-0.2,0.8) {\footnotesize HB};
		\draw[thick,rotate around={60:(-0.65,-0.2)}] (-0.65,-0.2) ellipse (0.55 and 0.15);
		\node[] at (-1.1,-0.1) {\footnotesize NB};
		\draw[thick,rotate around={30:(0.75,0.15)}] (0.75,0.15) ellipse (1.75 and 0.15);
		\node[] at (1.25,0) {\footnotesize FB};
		\end{scope}
		\draw[very thick,->] (zero) ++ (-0.15,0) -- (2,-2) node[below left] {Softness};
		\draw[very thick,->] (zero) ++ (0,-0.15) -- (-2,2) node[above left,rotate=90] {Hardness};
		\node[draw,thick,circle,fill=white] at (-2,-2) {C};
	\end{tikzpicture}
\label{fig:ns_z}
  \end{minipage}}
\hspace{6cm} % induce horizontal separation
\subfloat{%
  \begin{minipage}[c]{0.48\textwidth}
	\begin{tikzpicture}[xscale=2.5,yscale=1.25]
		\tikzset{lor/.style={thick, smooth, dotted, /pgf/fpu,/pgf/fpu/output format=fixed}}
		\coordinate (zero) at (-2,-2);
		\coordinate[right=2 of zero] (a1);
		\draw[very thick,->] (zero) ++ (-0.15,0) -- (2,-2) node[below left] {log $\nu$};
		\draw[very thick,->] (zero) ++ (0,-0.15) -- (-2,2) node[above left,rotate=90] {log $\nu \cdot$P};
		\node[draw,thick,circle,fill=white] at (-2,-2) {D};
		\draw[thick]  plot[smooth, tension=.7] coordinates {(-1.6,-2) (-1.3,0.0) (-1.1,0.2) (-0.9,0.4) (-0.65,0.1) (-0.5,0.1) (-0.1,-0.2) (0.35,0.35) (0.72,-1)(1.15,-2)};
		\clip (-2,-2) rectangle (2,2);
		\begin{scope}[shift={(-0.5,-0.25)}]
		%\draw[step=0.1cm,gray,very thin] (-2,-2) grid (2,2);
		%\draw[step=0.5cm,black,thin] (-2,-2) grid (2,2);
		\draw[lor] plot (\x-0.75,{1/(3.14*0.1*(1+((1000*\x)/0.1)^2))-2});
		\draw[lor] plot (\x,{3/(3.14*0.1*(1+((\x + 0.25 )/0.1)^2))-2});
		\draw[lor] plot (\x,{1/(3.14*0.1*(1+((\x)/0.1)^2))-3});
		\draw[lor] plot (\x,{1/(3.14*0.1*(1+((0.1*(2*\x-1.5))/0.1)^2))-2.5});
		\node[] at (-0.75,0.7) {\footnotesize $L_b$};
		\node[] at (-0.4,0.85) {\footnotesize $L_h$};
		\node[] at (0.1,0.5) {\footnotesize $L_{low}$};
		\node[] at (0.8,0.8) {\footnotesize $L_u$};
		\end{scope}
	\end{tikzpicture}
\label{fig:ns_ps}
  \end{minipage}}
\caption[Common diagrams for neutron stars]{%
\emph{Common diagrams for neutron stars} \spacedlowsmallcaps{upper left}~\spacedlowsmallcaps{(A)} A schematic representation of an energy spectrum showing the general shape of the main spectral components in terms of energy times flux \citep[see][]{lin2007evaluating}. The abbreviation $BB$ denotes a multicolour disk blackbody, $FL$ a fluorescent iron line, $BLBB$ the boundary layer blackbody and $PL$ a powerlaw, with the total of all components represented by the upper solid line.
\spacedlowsmallcaps{upper right}~\spacedlowsmallcaps{(B)}  A schematic \ac{ECC}~diagram for an atoll source, with various states denoted. Sources transition around the `C' shape, with starting from the lower right the \acf{UB}, \acf{LB}, \acf{LLB}, then the \acf{IS} and \acf{EIS} \citep{kleinwolt}.
\spacedlowsmallcaps{lower left (C)} A schematic Z source \ac{ECC}~diagram, showing the \acf{HB}, the \acf{NB} and the \acf{FB} \citep{kleinwolt}.
\spacedlowsmallcaps{lower right (D)} A simplified power spectrum of an atoll in the \ac{EIS} state showing various commonly fitted Lorentzian components. These components can rapidly change in number, amplitude, and width over time, while shifting in frequency. Sharply peaked components can emerge at the higher frequencies, referred to as kHz \acfp{QPO} \citep[see][]{kleinwolt,marieke}. 
	}\label{fig:ns_graphs}
\end{figure}
\end{landscape}